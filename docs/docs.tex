% Preamble
\documentclass{article}

% Packages
\usepackage{listings}
\lstset{language={[LaTeX]TeX}}

% Polish
\usepackage[T1]{fontenc}
\usepackage[polish]{babel}
\usepackage[utf8]{inputenc}
\usepackage{graphicx}
\usepackage{geometry}
\usepackage{xcolor}
\usepackage{listings}

% Macros
\newcommand{\hf}[1]{\frac{#1}{2}}

% Content of document
\begin{document}
\newgeometry{left=2cm,right=2cm, bottom=1.5cm, top=1.5cm}

% Top matter
\title{System do organizacji protestów}
\author{Marcin Martowicz}
\date{}
\maketitle

% Table of Contents (Toc)
\tableofcontents

\section{Model konceptualny}
\subsection{Diagram ER}
\includegraphics[width=500px]{Conceptual Model.png}

\subsection{Role}
\subsubsection{Activist}
\begin{itemize}
    \item{bierze udział w protestach}
    \item{składa sprawozdania z protestów}
    \item{za zasługi może awansować na pozycję organizatora}
\end{itemize}
\subsubsection{Organizer}
\begin{itemize}
    \item{przypisuje sobie rządowe akcje do obserwacji}
    \item{planuje protesty na temat akcji, do których jest przydzielony}
    \item{rejestruje ochroniarzy do systemu, sprawdzając poglądy polityczne wykrywaczem kłamstw}
    \item{wybiera ochroniarzy do swoich protestów}
    \item{może uczestniczyć w protestach jako aktywista}
\end{itemize}
\subsubsection{Guard}
\begin{itemize}
    \item{pilnuje porządku w czasie protestów}
    \item{na wstępie przechodzi test o poglądach politycznych}
    \item{zatrudnienie załatwia u organizatorów innym kanałem komunikacyjnym}
\end{itemize}

\subsection{Więzy}
\subsubsection{OrganizationMember}
\begin{itemize}
    \item{każdy członek musi być pełnoletni}
    \item{loginy muszą być unikalne}
    \item{hasło musi być niepuste}
\end{itemize}
\subsubsection{Participation}
\begin{itemize}
    \item{organizator musi uczestniczyć w swoim proteście}
\end{itemize}
\subsubsection{Report}
\begin{itemize}
    \item{ocena protestu musi być liczbą od $1$ do $10$}
    \item{treść sprawozdania musi być niepusta}
\end{itemize}
\subsubsection{Protest}
\begin{itemize}
    \item{musi być przynajmniej jeden boombox}
\end{itemize}
\subsubsection{GovernmentAction}
\begin{itemize}
    \item{tytuły akcji rządowych muszą być unikalne i niepuste}
\end{itemize}
\subsubsection{Guard}
\begin{itemize}
    \item{jeśli ochrania jakiś protest to musi popierać dany postulat}
    \item{musi umieć biegać z prędkością co najmniej $20$km/h}
    \item{musi ważyć co najmniej $80$kg}
\end{itemize}

\section{Model fizyczny}
Znajduje się w pliku create.sql.

\section{Obiecanka}
Zrobię aplikację w pythonie używając frameworku flask. Nie planuję zajmować się frontendem.

% that's not very pretty
% \section{API}
% \subsection{Dodawanie}
% \begin{description}
%     \item \textbf{register\_member}    [password] [name] [last\_name] [age] \\
%         Rejestracja nowego członka.
%     \item \textbf{register\_organizer} [password] [name] [last\_name] [age] [secret] \\
%         Rejestracja nowego organizatora.
%     \item \textbf{login} [password] \\
%         Logowanie użytkownika.
%     \item \textbf{observe\_action} [action\_name] \\
%         Przypisanie sobie akcji do obserwacji.
%     \item \textbf{create\_protest} [action\_id] [start\_time] [town] [geo\_coordinate] [boombox\_number] \\
%         Stworzenie protestu.
%     \item \textbf{participation} [protest\_id] [member\_id] \\
%         Zapisanie członka do protestu.
%     \item \textbf{report} [protest\_id] [protest\_id] [rating] [description] \\
%         Złożenie sprawozdania.
%     \item \textbf{guard} [name] [last\_name] [weight] [running\_speed] \\
%         Wprowadzenie ochroniarza do systemu.
%     \item \textbf{protection} [guard\_id] [protest\_id] \\
%         Zatrudnienie ochroniarza do protestu.
% \end{description}
% \subsection{Zapytania}
% \begin{description}
%     \item \textbf{participants} [protest\_id] \\
%         Zwraca członków zapisanych na protest.
%     \item \textbf{actions} [limit] \\
%         Zwraca do [limit] akcji wraz z ilością przypisanych protestów. Posortowane po ilości protestów.
%     \item \textbf{participants\_stats} [limit] \\
%         Zwraca do [limit] członków wraz z ilością protestów, w których brali udział, liczbą ich sprawozdań i sumaryczną długością sprawozdań. Posortowane malejąco po długości.
%     \item \textbf{organizer\_stats} [limit] \\
%         Zwraca do [limit] organizatorów wraz z ilością przypisanych protestów i średnią oceną ze sprawozdań. Posortowane malejąco po ocenie.
%     \item \textbf{find\_protest} [point] [start\_time] [end\_time] [limit] \\
%         Zwraca do [limit] protestów, które odbywają się w danym przedziale czasowym. Posortowane po dystansie do podanego punktu.
%     \item \textbf{most\_profitable\_protests} [guard\_id] [start\_time] [end\_time] [limit] \\
%         Zwraca do [limit] protestów, które odbywają się w danym przedziale czasowym i dany ochroniarz nie jest do nich przypisany. Posortowane po liczbie boomboxów, które przypadną mu do ochrony (ochroniarze dzielą się po równo).
%     \item \textbf{indirect\_friends} [member\_id] \\
%         Po wspólnym proteście wszyscy członkowie zostają kolegami. Zapytanie zwraca bezpośrednich i pośrednich znajomych danego członka.
% \end{description}

% \section{Uruchomienie}
% Nadanie obecnemu użytkownikowi prawa do tworzenia bazy danych.
% \begin{lstlisting}[language=Bash]
% sudo su postgres
% psql -c 'ALTER ROLE <uzytownik> with CREATEDB;'
% \end{lstlisting}

\end{document}
% Typesetting is off

