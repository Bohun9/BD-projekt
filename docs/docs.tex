% Preamble
\documentclass{article}

% Packages
\usepackage{listings}
\lstset{language={[LaTeX]TeX}}

% Polish
\usepackage[T1]{fontenc}
\usepackage[polish]{babel}
\usepackage[utf8]{inputenc}
\usepackage{graphicx}
\usepackage{geometry}
\usepackage{xcolor}
\usepackage{listings}

% Macros
\newcommand{\hf}[1]{\frac{#1}{2}}

% Content of document
\begin{document}
\newgeometry{left=2cm,right=2cm, bottom=1.5cm, top=1.5cm}

% Top matter
\title{System do organizacji protestów}
\author{Marcin Martowicz}
\date{}
\maketitle

% Table of Contents (Toc)
\tableofcontents

\section{Model konceptualny}
\subsection{Diagram ER}
\includegraphics[width=500px]{Conceptual Model.png}

\subsection{Role}
\subsubsection{Activist}
\begin{itemize}
    \item{bierze udział w protestach}
    \item{składa sprawozdania z protestów}
    \item{za zasługi może awansować na pozycję organizatora}
\end{itemize}
\subsubsection{Organizer}
\begin{itemize}
    \item{przypisuje sobie rządowe akcje do obserwacji}
    \item{planuje protesty na temat akcji, do których jest przydzielony}
    \item{rejestruje ochroniarzy do systemu, sprawdzając poglądy polityczne wykrywaczem kłamstw}
    \item{wybiera ochroniarzy do swoich protestów}
    \item{może uczestniczyć w protestach jako aktywista}
\end{itemize}
\subsubsection{Guard}
\begin{itemize}
    \item{pilnuje porządku w czasie protestów}
    \item{na wstępie przechodzi test o poglądach politycznych}
    \item{zatrudnienie załatwia u organizatorów innym kanałem komunikacyjnym}
\end{itemize}

\subsection{Więzy}
\subsubsection{OrganizationMember}
\begin{itemize}
    \item{każdy członek musi być pełnoletni}
    \item{loginy muszą być unikalne}
    \item{hasło musi być niepuste}
\end{itemize}
\subsubsection{Participation}
\begin{itemize}
    \item{organizator musi uczestniczyć w swoim proteście}
\end{itemize}
\subsubsection{Report}
\begin{itemize}
    \item{ocena protestu musi być liczbą od $1$ do $10$}
    \item{treść sprawozdania musi być niepusta}
    \item{osoba składająca raport musi być zapisana na protest (relacja w telewizji może być zmanipulowana)}
    \item{osoba składająca raport nie może wysłać drugiego sprawozdania do tego samego protestu}
\end{itemize}
\subsubsection{Protest}
\begin{itemize}
    \item{musi być przynajmniej jeden boombox}
\end{itemize}
\subsubsection{GovernmentAction}
\begin{itemize}
    \item{tytuły akcji rządowych muszą być unikalne i niepuste}
\end{itemize}
\subsubsection{Guard}
\begin{itemize}
    \item{jeśli ochrania jakiś protest to musi popierać dany postulat}
    \item{musi umieć biegać z prędkością co najmniej $20$km/h}
    \item{musi ważyć co najmniej $80$kg}
\end{itemize}

\section{Model fizyczny}
Znajduje się w pliku \texttt{protest/schema.sql}.

\section{API}
\subsection{Dodawanie}
\begin{description}
    \item \texttt{register\_member} \texttt{[login]} \texttt{[password]} \texttt{[name]} \texttt{[last\_name]} \texttt{[age]} \\
        Rejestracja nowego członka.
    \item \texttt{register\_organizer} \texttt{[login]} \texttt{[password]} \texttt{[name]} \texttt{[last\_name]} \texttt{[age]} \texttt{[secret]} \\
        Rejestracja nowego organizatora.
    \item \texttt{login} \texttt{[login]} \texttt{[password]} \\
        Logowanie użytkownika.
    \item \texttt{logout} \\
        Wylogowanie.
    \item \texttt{observe\_action} \texttt{[action\_name]} \\
        Utworzenie rządowej akcji i przypisanie jej do obserwacji.
    \item \texttt{add\_protest} \texttt{[action\_id]} \texttt{[start\_time]} \texttt{[town]} \texttt{[coordinates]} \texttt{[boombox\_number]} \\
        Stworzenie protestu.
    \item \texttt{add\_participation} \texttt{[protest\_id]} \texttt{[member\_id]} \\
        Zapisanie członka do protestu.
    \item \texttt{add\_report} \texttt{[protest\_id]} \texttt{[rating]} \texttt{[description]} \\
        Złożenie sprawozdania.
    \item \texttt{register\_guard} \texttt{[name]} \texttt{[last\_name]} \texttt{[weight]} \texttt{[running\_speed]} \\
        Wprowadzenie ochroniarza do systemu.
    \item \texttt{add\_worldview} \texttt{[guard\_id]} \texttt{[action\_id]} \\
        Wprowadzenie informacji, że dany ochroniarz popiera dane działanie rządowe.
    \item \texttt{add\_protection} \texttt{[guard\_id]} \texttt{[protest\_id]} \\
        Zatrudnienie ochroniarza do protestu.
\end{description}
\subsection{Zapytania}
\begin{description}
    \item \texttt{participants} \texttt{[protest\_id]} \\
        Zwraca członków zapisanych na protest.
    \item \texttt{actions\_stats} \texttt{[limit]} \\
        Zwraca akcje wraz z ilością przypisanych protestów i sumaryczną liczbą różnych osób zaangażowanych w protesty. Posortowane po ilości protestów.
    \item \texttt{participants\_stats} \texttt{[limit]} \\
        Zwraca członków wraz z ilością protestów, w których brali udział, liczbą ich sprawozdań i sumaryczną długością sprawozdań. Posortowane malejąco po długości.
    \item \texttt{organizer\_stats} \texttt{[limit]} \\
        Zwraca organizatorów wraz z ilością przypisanych protestów i średnią oceną ze sprawozdań. Posortowane malejąco po ocenie.
    \item \texttt{find\_closest\_protests} \texttt{[point]} \texttt{[start\_time]} \texttt{[end\_time]} \texttt{[limit]} \\
        Zwraca protesty, które odbywają się w danym przedziale czasowym. Posortowane po dystansie do podanego punktu.
    \item \texttt{find\_profitable\_protests} \texttt{[guard\_id]} \texttt{[start\_time]} \texttt{[end\_time]} \texttt{[limit]} \\
        Zwraca protesty, które odbywają się w danym przedziale czasowym i dany ochroniarz nie jest do nich przypisany. Posortowane po liczbie boomboxów, które przypadną mu do ochrony (ochroniarze dzielą się po równo).
    \item \texttt{indirect\_friends} \texttt{[member\_id]} \\
        Po wspólnym proteście wszyscy członkowie zostają kolegami. Zapytanie zwraca bezpośrednich i pośrednich znajomych danego członka.
\end{description}

\section{Obiecanka}
Użyję języka \texttt{python} i frameworku \texttt{flask}. Aplikacja będzie odpowiadała na żądania HTTP surowym jsonem bez frontendu.
API powinno być mniej więcej takie jak wyżej.

\end{document}
% Typesetting is off

